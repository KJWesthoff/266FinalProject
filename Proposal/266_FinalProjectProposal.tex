%%%%%%%%%%%%%%%%%%%%%%%%%%%%%%%%%%%%%%%%%
% Journal Article
% LaTeX Template
% Version 2.0 (February 7, 2023)
%
% This template originates from:
% https://www.LaTeXTemplates.com
%
% Author:
% Vel (vel@latextemplates.com)
%
% License:
% CC BY-NC-SA 4.0 (https://creativecommons.org/licenses/by-nc-sa/4.0/)
%
% NOTE: The bibliography needs to be compiled using the biber engine.
%
%%%%%%%%%%%%%%%%%%%%%%%%%%%%%%%%%%%%%%%%%

%----------------------------------------------------------------------------------------
%	PACKAGES AND OTHER DOCUMENT CONFIGURATIONS
%----------------------------------------------------------------------------------------

\documentclass[
	letterpaper, % Paper size, use either a4paper or letterpaper
	12pt, % Default font size, can also use 11pt or 12pt, although this is not recommended
	unnumberedsections, % Comment to enable section numbering
	twoside, % Two side traditional mode where headers and footers change between odd and even pages, comment this option to make them fixed
]{LTJournalArticle}

\addbibresource{bibliography.bib} % BibLaTeX bibliography file

\runninghead{Project Proposal} % A shortened article title to appear in the running head, leave this command empty for no running head

\footertext{\textit{Project Proposal, Final Project} (MICS/DATSCI 266, Summer 2025)} % Text to appear in the footer, leave this command empty for no footer text

\setcounter{page}{1} % The page number of the first page, set this to a higher number if the article is to be part of an issue or larger work

%----------------------------------------------------------------------------------------
%	TITLE SECTION
%----------------------------------------------------------------------------------------

\usepackage[title,toc,titletoc]{appendix}
\usepackage{titlesec}
\usepackage{lscape}
\usepackage{fontawesome}


\title{Manipulation Detection \\ Project Proposal \\ 266 Final Project} % Article title, use manual lines breaks (\\) to beautify the layout}

% Authors are listed in a comma-separated list with superscript numbers indicating affiliations
% \thanks{} is used for any text that should be placed in a footnote on the first page, such as the corresponding author's email, journal acceptance dates, a copyright/license notice, keywords, etc
\author{
	Karl-Johan Westhoff \\
	email \href{mailto:kjwesthoff@berkeley.edu}{kjwesthoff@berkeley.edu}
}

% Affiliations are output in the \date{} command
\date{UC Berkeley School of Information \\
MIDS Course 266 Summer 2025 Section 2 (Natalie Ahn) \\
}

% % Full-width abstract
% \renewcommand{\maketitlehookd}{%
% 	\begin{abstract}
% 		\noindent Lorem ipsum dolor sit amet,rta porttitor.
% 	\end{abstract}
% }

%----------------------------------------------------------------------------------------
\setcounter{tocdepth}{5}
\setcounter{secnumdepth}{5}
\usepackage[title]{appendix}

\begin{document}
\maketitle % Output the title section



%----------------------------------------------------------------------------------------
%	ARTICLE CONTENTS
%----------------------------------------------------------------------------------------


\section{Introduction}
The human factor is increasing in relation to cyber attacks. The 2024 Verizon DBIR report \cite{verizon2024dbir}.mentions that 68\% of all cyber breaches involved the human element with phishing being a major contributor LLM's can now generate perfect and very convincing email text payloads, making current detection systems based on bag-of-words models ineffective\\
The purpose of this project is to develop a model that can detect if an email is manipulative and tries to make you take actions that are not in your best interest.

\section{Datasets}

\subsection{Training Dataset}
The model will be trained on a dataset generated by Wang et al. \cite{MentalManip} which is based on 4000 labeled dialogues from films.\\

\subsection{Evaluation Dataset}
We will use data sets with labeled phishing emails in combination with results from previous models.

\section{Methods}
We will build an inference model that can detect manipulated emails based on a deep neural network with transformer architecture.

\section{Evaluation}
Our main interest is to investigate if the model can extend existing phishing detection systems by detecting manipulating language in the emails. We will to look at false negative results from previous models, to see if the detection of manipulative text captures emails that were previously missed.








%----------------------------------------------------------------------------------------
%	 REFERENCES
%----------------------------------------------------------------------------------------
\clearpage
\printbibliography % Output the bibliography
%----------------------------------------------------------------------------------------

%----------------------------------------------------------------------------------------
%	 Appendices
%---------------------------------------------------------------------------------------


%\begin{appendices}
%	\onecolumn
%
%	\section{Appendix}
%
%
\end{document}
